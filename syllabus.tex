%% LyX 1.6.8 created this file.  For more info, see http://www.lyx.org/.
%% Do not edit unless you really know what you are doing.
\documentclass[english]{article}
\usepackage[T1]{fontenc}
\usepackage[latin9]{inputenc}
\usepackage{array}

\makeatletter

%%%%%%%%%%%%%%%%%%%%%%%%%%%%%% LyX specific LaTeX commands.
%% Because html converters don't know tabularnewline
\providecommand{\tabularnewline}{\\}

%%%%%%%%%%%%%%%%%%%%%%%%%%%%%% User specified LaTeX commands.
\usepackage{layout}
\usepackage[parfill]{parskip}
\usepackage[margin=1in]{geometry}
\geometry{letterpaper}

\usepackage{fancyhdr}
\setlength{\headheight}{15.2pt}
\pagestyle{fancy}
\lhead{calopensource@gmail.com}
\chead{Course Syllabus}
\rhead{Monday 6-8pm}

\usepackage{graphicx} %loads the graphicx.sty package 
\usepackage{epstopdf} %loads the epstopdf.sty package 
          % to automatically convert eps files to pdf format 

% the next line converts tif files to png
\DeclareGraphicsRule{.tif}{png}{.png}{%
       `convert #1 `dirname #1`/`basename #1 .tif`.png}

\usepackage{array}
\setlength{\extrarowheight}{5pt}

\makeatother

\usepackage{babel}

\begin{document}

\title{Introduction to Open Source Software}


\author{\textbf{Facilitators:} Seshadri Mahalingam, Kevin Porter, Anirudh
Todi\textbf{}\\
\textbf{Intructor of Record: }David Wagner}


\date{Fall 2011}

\maketitle

\part*{Course Syllabus}


\section{Course Goals}

This course aims to provide a community and lecture series for students
to learn the skills they will need to contribute to open-source projects,
and large scale software engineering projects in general. If you have
ever been interested in joining an open source project but felt that
the barrier to entry was too high, this DeCal is for you!\\
\\
Projects will be worked on in groups of two or more people. 


\section{Suggested Prerequisites}

You\textquoteright{}ll get the most of out of this class if you\textquoteright{}re
familiar with at least one language used for open source development
(Python, Perl, PHP, Java, C++, C, Ruby and many more) to a 61B level.
This course will tend towards technical contribution, but those interested
in contributing in design or user experience areas are welcome too.


\section{Grading Policy}

The course will be offered only on a P/NP basis. The following points
dictate requirements to be met to pass the course.


\subsection{Attendance}

The weekly DeCal meetings are mandatory. Up to two absences will be
permitted for a passing grade. Please email or otherwise notify us
ahead of time if you need to miss a meeting, and we\textquoteright{}ll
let you know how you can make it up.


\subsection{Contribution}

Contribute to your chosen project through patch submission, UX improvement,
or improving code documentation. One of the primary objectives of
the class is to help you attain a familiarity with open source projects
that prompts contribution to the community. As a part of this {}``assignment''
post a note on your project page on the class wiki with your name,
a link to your commit and a description of what you did. Each member
of a project group will be required to have a contribution of their
own.


\subsection{Group Presentation}

Make a presentation to the class summarizing your work on an open
source project. Each project group will be expected to present for
approximately 15 minutes, followed by an up to 5 minute question-and-answer
session. Your presentation should contain or address the following
points:
\begin{itemize}
\item A description of the open source project, detailing its function and
licencing%
\footnote{American English uses {}``licensing'' instead.%
} structure.
\item The version control, bug tracking, wiki and documentation systems
used in the project.
\item The bug or feature each project group member contributed to resolving/implementing.
\end{itemize}

\subsection{Assignments}

Assignments may be handed out on certain weeks to help progress. These
will not be graded. However, the assignments will tend to be tasks
such as setting up development tools or wiki pages, or downloading
and compiling source code. Their intention is to keep you on track
towards conributing to your project and making the group presentation.


\section{Academic Honesty}

While the class is inherently collaborative, the course will require
individual contributions. Any submission for this class to which you
have added your name must be your original work. The open source community
is built on mutual trust between developers, and considers plagiarism
and stealing credit to be highly offensive. Observe restrictions imposed
by the chosen licence of a software project.


\section{Learning Goals}

Students will learn skills for working with large scale projects.
These will include: 
\begin{itemize}
\item source control systems, bug trackers and software development wikis
\item code review systems and how to write readable code 
\item how to write and submit a clearly documented patch 
\item code design, using automated tools, tricks of the trade
\end{itemize}

\section{Class Format}

Meetings will be held on Mondays from 6-8PM in 320 Soda, and will
consist of the following format: 
\begin{itemize}
\item Lecture -- We\textquoteright{}ll present material relating to the
class themes for that week or frame case studies for class discussion
\item Group Activity -- We\textquoteright{}ll frame specific activities
relating to current course themes: for example, during discussion
of code review approaches, a group activity could involve students
reviewing each other\textquoteright{}s code
\item Collaborative hacking -- Students will work on their projects, asking
each other or the facilitators for assistance as needed.
\end{itemize}
Discussion between students is highly encouraged, and will be facilitated
by a mailing list.


\section{Draft Lecture Schedule}

\begin{tabular}{|c|c|>{\centering}p{7cm}|}
\hline 
\textbf{Week} & \textbf{Topic(s)} & \textbf{Description}\tabularnewline
\hline
\hline 
1 & Introduction & \tabularnewline
\hline 
2 & Where can I find open source projects? & Introduction to the open source community through project listings,
social coding web sites and specific projects.\tabularnewline
\hline 
3 & Version Control, Project selections & \tabularnewline
\hline 
4 & Guest Lecture - Licencing & \tabularnewline
\hline 
5 & Bug tracking, Wikis & \tabularnewline
\hline 
6 & Guest Lecture - Talking about code & \tabularnewline
\hline 
7 & Diff tools, Patches, Hacking & \tabularnewline
\hline 
8 & Guest Lecture - Open Source Project & \tabularnewline
\hline 
9 & Presentations & \tabularnewline
\hline 
10 & Presentations & \tabularnewline
\hline 
11 & Presentations & \tabularnewline
\hline 
12 & Presentations, Closing Remarks & \tabularnewline
\hline
\end{tabular}


\subsection{Getting Started}
\begin{itemize}
\item Choosing a project -- feature ideas -- bootstrapped from Google Summer
of Code. 
\item Using version control (distributed vs centralized) 
\item IRC / Wiki use 
\item Code Review -- how to talk about your code 
\item How to read code -- assignment idea: read a snippet of open-source
code and present what it does
\item Building your project
\end{itemize}

\subsection{Making Contributions}
\begin{itemize}
\item Submitting bug reports -- bug tracker tools (Launchpad, bugzilla,
trac and others) 
\item Using diff tools, submitting patches 
\item Reproducible errors -- the art of a good test case
\item Documenting and commenting your code 
\item Licensing and why it matters -- possibly guest discussion with Prof.
Carver @ iSchool 

\begin{itemize}
\item GPL-what?
\item Using open source code correctly in a commercial/production setting
\end{itemize}
\end{itemize}

\subsection{Case Studies}
\begin{itemize}
\item Ubuntu -- Ayatana -- The life-cycle of a paper-cut bug
\item SQLite -- an example of readable, well-documented code 
\item Google Chrome and Webkit -- guest discussion with a Webkit contributer 
\item More TBD
\end{itemize}

\section{Application}

To apply for enrollment into the decal, visit the course website (at
the top of this syllabus), and click on \textquotedblleft{}Application
to join\textquotedblright{}. Fill out and submit the application by
11:59 PM on Wednesday, Feb 2nd to be considered for enrollment.
\end{document}
